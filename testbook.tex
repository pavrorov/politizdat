\documentclass[draft,citemore,headerson,openany,sectoc]{politizdat}
\usepackage[main=russian]{babel}
\usepackage{fontspec}
\defaultfontfeatures{Ligatures=TeX}
\setmainfont{Latin}
\newfontfamily\russianfont{Latin}%
\newfontfamily\germanfont{OldStandard}%

\usepackage{amssymb} % для звёздочки в титуле
\usepackage[bottom]{footmisc}
\usepackage{hyphenat}
\usepackage{url}\urlstyle{same}

% В шрифте Latin пока нет многоточия и ударения
\let\ldotscmd\ldots
\renewcommand{\ldots}{%
  {\small\germanfont\ldotscmd}%
}
\usepackage{newunicodechar}
\newunicodechar{́}{{\germanfont ́}}
\newunicodechar{…}{{\small\germanfont …}}

\bibliography[labelnumber=false]{bib/marx-engels,bib/lenin,bib/stalin,bib/ikki,bib/misc,bib/philosophy,bib/plehanov,bib/s-d,bib/revision,bib/reaction}


\begin{document}

\begin{centertitlingpage}
  \begin{center}
  \vspace*{20pt}
  \OnehalfSpacing
  
  \fontsize{32pt}{36pt}\selectfont
  КНИГА
  
  \medskip\huge
  ДЛЯ ПРИМЕРА
  
  \vspace{\baselineskip}
  $\bigstar$
  
  \vspace{\baselineskip}
  {\itshape Демонстрирующая работу пакета \qq{Политиздат}}
  
  \vfill\normalsize
  \qq{ПОЛИТИЗДАТ}~\textbullet~2018
\end{center}

\end{centertitlingpage}

\setcounter{tocdepth}{3}

\frontmatter*
\sloppybottom
\chapter{Предисловие}

Пример предисловия.


\mainmatter
\sloppybottom
\chapter{Первая глава}

\section{Раздел первый}

Пример главы.

\section{Раздел второй}

Продолжение примера главы.

\chapter{Вторая глава}

\section{Раздел первый}

Пример ещё одной главы.

\section{Раздел второй}

Продолжение примера главы.


\backmatter
\tableofcontents*
\end{document}
